The Large Hadron Collider (\LHC) is the largest and highest energy particle accelerator
in the world, designed to collide protons with an unprecedented centre-of-mass
energy of \unit{14}{\TeV} and instantaneous luminosity of \highL. In addition, the
\LHC has a heavy ion collision programme, aiming to collide lead nuclei with a
centre-of-mass energy of \unit{5.5}{\TeV}. In the early phase of operation, the
proton-proton programme at the \LHC has been operating with reduced centre-of-mass
energies of up to \unit{7}{\TeV}; these nevertheless represent the highest energy
collisions that have yet been attained in a particle accelerator.

The accelerator was built by the European Organisation for Nuclear Research
(\CERN) and is situated in a \unit{27}{\kilo\metre} long circular tunnel
spanning the Swiss-French border near Geneva. A superconducting helium-cooled
dipole magnet system, operating at \unit{8.3}{\tesla} is used to guide protons or
lead nuclei around this ring.

The \LHC was built with the aim of testing the current understanding of high
energy physics, with its programme ranging from precise measurements of Standard
Model parameters through to searches for new physics phenomena and investigations of the
properties of strongly interacting matter at extreme energy densities. \ATLAS (A
Toroidal LHC ApparatuS) is, together with CMS (Compact Muon Solenoid), one of two
general purpose detector experiments which have been collecting data at the \LHC.
Ideally suited to explore the \TeV energy domain, \ATLAS will play an important role in
the possible resolution of these fundamental questions in particle physics.

In this thesis, a number of separate analyses are presented, each aiming to probe our
understanding of \QCD in this new energy regime, with $\rootS = \unit{7}{\TeV}$.
Differential \xs{s} of inclusive jets and \dijet{s} are performed across two
orders of magnitude in jet transverse momentum and \dijet mass and are compared
to next-to-leading order theoretical predictions. Radiation between \dijet{s} is
examined as a possible means of discriminating between DGLAP and \BFKL-like
parton evolution schemes.

Finally, a more technical contribution to this thesis is a technique for
ascertaining the uncertainty on the jet energy scale through intercalibration
between jets in different regions of the detector.
