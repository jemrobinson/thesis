A variety of jet measurements are made using data collected during the first
year of \unit{7}{\TeV} proton-proton collisions from the general-purpose \ATLAS
experiment at the \LHC. The data are compared to leading-order and next-to-leading
order \MC predictions, which have been interfaced with a parton shower, as well
as to next-to-leading order perturbative \QCD calculations, which have been
corrected for soft effects. In each case, state-of-the-art jet algorithms are
used, allowing for a better comparison between data and theory.

Two distinct types of analysis are presented in this thesis, measurements of jet
\xs{s} and investigations of \QCD emissions in \dijet systems. Double
differential jet \xs{s}, as a function of jet transverse momentum and rapidity
or \dijet mass and rapidity separation, provide an exacting test of \QCD across
several orders of magnitude. The study of \QCD radiation in \dijet systems is
performed by vetoing on any \QCD activity above a veto scale, \Qnought much
greater than $\Lambda_{QCD}$, permitting the study of a wide range of
perturbative \QCD phenomena. Less well understood areas of phase space, in which
standard event generators have large theoretical uncertainties, can be probed in
both widely separated \dijet systems and in those with average transverse
momenta much greater than the veto scale.
