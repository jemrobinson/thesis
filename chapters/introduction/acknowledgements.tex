\DeclareRobustCommand{\personDG}{Dag Gillberg\xspace}
\DeclareRobustCommand{\personAP}{Andrew Pilkington\xspace}
\DeclareRobustCommand{\personJK}{Justin Keung\xspace}
\DeclareRobustCommand{\personEF}{Eric Feng\xspace}
\DeclareRobustCommand{\personCM}{Christopher Meyer\xspace}
\DeclareRobustCommand{\personMC}{Mario Campanelli\xspace}
\DeclareRobustCommand{\personPB}{Pauline Bernat\xspace}
\DeclareRobustCommand{\personGB}{Gareth Brown\xspace}
\DeclareRobustCommand{\personAD}{Adam Davison\xspace}
\DeclareRobustCommand{\personJM}{James Monk\xspace}

Contemporary particle physics experiments are, in general, conducted by large
collaborations. Many thousands of people work on the experiments based at the
\LHC and, as such, it is necessary to make clear which parts of this thesis are
my own work.

Over the last century, a great deal of research has been performed into the
fundamental structure of matter, much of which is condensed into the Standard
Model of particle physics; some of the relevant parts of this theoretical
framework are briefly discussed in \ChapterRef{chap:bg-theory}. Equally, the
\ATLAS and \LHC collaborations have worked over the last twenty years to design
and build the physical machinery without which the measurements described in
this thesis could not have been made. A short description of the layout and
workings of some parts of the \ATLAS detector is made in
\ChapterRef{chap:detector} although, by necessity, this cannot do justice to the
huge complexity of this machine. The \ATLAS collaboration also provides a data
transfer system and software framework, within which context all of the
following results were obtained.

The contents of \ChapterRef{chap:eta-intercalibration} are strongly based on
technical work, performed in collaboration with \personDG and \personAP among
others, which was used as one of the inputs to the determination of the \ATLAS
jet energy scale uncertainty. A novel soft radiation correction is also
presented here; accordingly, all of the plots and numbers here are my own work
and, although consistent with, are not identical to the official results.

In \ChapterRef{chap:forward-inclusive}, my contribution was primarily to the
determination of the \xs in the forward region, $2.8 \leq \absRap < 4.4$,
particularly in developing the trigger strategy used here, together with
\personDG and \personJK. In \ChapterRef{chap:dijets}, I developed and validated
the two-jet trigger strategy, in conjunction with \personEF and \personCM, based
on an idea from \personMC. All the final \xs{s} shown in both of these chapters
represent official \ATLAS results.

The contents of \ChapterRef{chap:gbj} and \ChapterRef{chap:azimuthal-decorrelation}
represent an amalgam of published and ongoing research into \QCD radiation
phenomena. In the first instance, my contribution was primarily in determining
background contributions and the stability of the final distributions under
various changes to the selection criteria. Final plots shown here are again,
official \ATLAS results. In the later work, I transferred and revalidated the
two-jet trigger strategy for a new set of variables, while also providing the
bin-by-bin unfolded data. All plots shown in this chapter represent my own work.
In both cases, I am heavily indebted to the other contributors to these analyses,
including \personAP, \personPB and \personGB.

The work presented in this thesis would not have been possible without the
assistance and encouragement of the High Energy Physics group at \UCL,
particularly \personAD and \personJM, who have shown a lot of patience despite
my frequent physics and technical questions. Obviously, the \ATLAS
Collaboration, and the Jet and Missing-\ET group in particular, have been
instrumental in allowing me to complete this work; I owe particular thanks to
\personAP and \personDG in this regard. It should, of course, be understood that
all of the work in this thesis was performed with the help and guidance of my
supervisor, \personMC.

Finally, thanks are due to my parents and my sister for the love and support
that they have given me, not just over the past three years but throughout my
life.
