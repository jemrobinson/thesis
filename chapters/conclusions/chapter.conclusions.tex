\chapter{Conclusions}
\label{chap:conclusions}

\chapterquote{I conclude that there is no thing constantly observable in nature,
which will not always bring some light with it, and lead us farther into the knowledge
of her ways of working.}{John Locke}

\QCD studies are among the first measurements that can be made at hadron colliders;
at the \LHC jet production is the dominant high \pT process. Jet physics observables
such as \xs{s} are important for improving our understanding of the strong interaction,
particularly through providing measurements of \alphaS at a range of scales which
can then be used to constrain PDF fits.

Accurately calibrated jets are essential in order to make these measurements; \dijet
intercalibration is one of the in-situ methods used to validate the jet energy scale,
which is primarily derived from \MC and test beam data. Results from intercalibration
indicate that the relative response to both high \pT and central jets is well understood.
Maximal disagreement occurs for low \pT, high rapidity jets. As well as providing a cross-check for the jet energy scale
in the central region, intercalibration also provides one of the means by which
it can be extended into the forward region.

Jet and \dijet \xs measurements conducted across the full accessible kinematic range
provide information that has never been available before, particularly in the forward
region, which previous hadron colliders have not been able to explore with such
precision. In those regions of phase space in which experimental and theoretical
uncertainties are similar in size, these measurements provide some discriminating
power between different theoretical models. Overall, after corrections for non-perturbative effects, NLO perturbative \QCD predictions,
agree with the measurements across seven orders of magnitude in \xs. The greatest
disagreements are seen at large values of jet transverse momentum and \dijet invariant
mass, where the theoretical predictions for the \xs{s} tend to be larger than the
measured values. These measurements probe, and may help to constrain, the previously unexplored area of parton distribution
functions at large $x$ and high $Q^2$, representing one of the most comprehensive
tests of \QCD ever performed.

In \dijet events, the extent of hadronic activity in the rapidity interval between
the jets can be studied with the use of a central jet veto. When considering the
fraction of events which have no activity above the veto scale in this region, most
experimental uncertainties cancel. Measurements of this ratio show the expected
behaviour of a reduction of gap events as the \dijet system becomes harder or more
widely separated in rapidity. Good agreement is found with both leading-order \MC
simulations and NLO predictions interfaced with a parton shower. No evidence was
found that a \BFKL-like description of parton evolution would provide an improved
agreement between data and theoretical predictions. This data can be used to constrain
the event generator modelling of \QCD radiation between widely separated jets. Such
a constraint is useful for the current Higgs-plus-two-jet searches and also for
any future measurements sensitive to higher order \QCD emissions.
