%% The "\appendix" call has already been made in the declaration
%% of the "appendices" environment (see thesis.tex).
\chapter{Combining Multiple Triggers}
\label{chap:appendix:trigger_efficiencies}

\chapterquote{It should be done with the same degree of alacrity and nonchalance 
              that you would display in authorising a highly intelligent trained
              bear to remove your appendix.}
{Daniel S. Greenberg}

Consider the case in which there are two triggers of interest, one for each of the
two leading jets. The case in which the leading jet passes the appropriate trigger
is denoted $T_{10}$ and the case where the second jet passes its trigger as $T_{01}$,
while the case in which both do is labelled $T_{11}$. In the absence of any prescales
or inefficiencies, we would expect all $N$ of our events to be in category $T_{11}$,
since our trigger boundaries are chosen to ensure that all jets fall on the trigger
efficiency plateau. If the amount of data taken corresponds to a luminosity $\luminosity_{true}$
then we have a \xs of $\sigma_{true} = N / \luminosity_{true}$.

\section{Correcting for Inefficiency in the Absence of Prescale}
Taking specific detector effects into account, which may lower the efficiencies
of the leading and subleading triggers to $e_L$ and $e_S$ respectively, we can see that the distribution
of events will be as summarised in \TableRef{tab:appendix:trig_inefficiencies}.

\begin{table}
\begin{center}
  \begin{tabular}{ c c }
    Trigger Category & Number of events      \\
    \midrule
    $T_{00}$         & $N(1-e_L)(1-e_S)$ \\
    $T_{10}$         & $N e_L (1-e_S)$   \\ 
    $T_{01}$         & $N e_S (1-e_L)$   \\
    $T_{11}$         & $N e_L e_S$       \\       
  \end{tabular}
  \caption{Number of events in each trigger category after allowing for trigger
           inefficiencies}
  \label{tab:appendix:trig_inefficiencies}
\end{center}
\end{table}

Obviously events in the category $T_{00}$ are not recorded, but we still see
$N_{visible} = N(e_L + e_S - e_L e_S)$ events. Multiplying the luminosity by an overall
efficiency correction of $e_{eff} = e_L + e_S - e_L e_S$ allows us to recover the
correct \xs:

\begin{equation}
  \sigma_{corrected} = \frac{N_{visible}}{e_{eff} \luminosity_{true}} = \frac{N(e_L + e_S - e_L e_S)}{(e_L + e_S - e_L e_S) \luminosity_{true}} = \sigma_{true}
\end{equation}

\section{Correcting for Prescale in the Absence of Inefficiency}
\label{sec:appendix:prescale_only}
Allowing now for prescales of $P_L$ and $P_S$ for the leading and subleading triggers
respectively, we can see that the distribution of events will be as summarised
in \TableRef{tab:appendix:trig_prescales}.

\begin{table}
\begin{center}
  \begin{tabular}{ c c }
    Trigger Category & Number of events      \\
    \midrule
    $T_{00}$         & $N(1-1/P_L)(1-1/P_S)$ \\
    $T_{10}$         & $N/P_L (1-1/P_S)$     \\ 
    $T_{01}$         & $N/P_S (1-1/P_L)$     \\
    $T_{11}$         & $N/(P_L P_S)$          \\       
  \end{tabular}
  \caption{Number of events in each trigger category after allowing for trigger
           prescales}
  \label{tab:appendix:trig_prescales}
\end{center}
\end{table}

Analogously to the trigger efficiency case, we can apply a prescale correction of
$1/P_{eff} = 1/P_L + 1/P_S - 1/(P_L P_S)$ to the luminosity to recover the correct
\xs. 

\begin{equation}
  \sigma_{corrected} = \frac{N_{visible}}{\luminosity_{true}/P_{eff}} = \frac{N(1/P_L + 1/P_S - 1/(P_L P_S))}{\luminosity_{true} (1/P_L + 1/P_S - 1/(P_L P_S))} = \sigma_{true}
\end{equation}

In general, statistical precision will be improved by instead %the case when $N$ is large
dividing events into separately weighted trigger categories, based on the trigger
decision that would have been taken in the absence of prescale. This approach
is discussed in more detail in \SectionRef{sec:forward-inclusive:transition_triggers}.

\section{Correcting for Inefficiency and Prescale Simultaneously}
In the case in which efficiencies of $e_L$ and $e_S$ are present together with
prescales of $P_L$ and $P_S$, recovering the correct \xs becomes more complicated.

If we could recover the prescale-only (PO) numbers, then recovering the correct
\xs would be a matter of following the prescription from \SectionRef{sec:appendix:prescale_only}.
In fact, using the simplified approach presented there, knowing only $N^{PO}_{visible} = N^{PO}_{11} + N^{PO}_{10} + N^{PO}_{01}$
would be sufficient.

Consider the relationships, shown in \EquationRef{eq:appendix:relationships},
between the final observed (F) numbers of events observed and the numbers after
prescale-only, where $L$ denotes events passing the leading jet trigger and $S$
denotes those passing the second trigger, while $LS$ denotes those passing both triggers.

\begin{equation}
\begin{split}
  N^{F}_{L}  &= N^{PO}_{L} e_L \\
  N^{F}_{S}  &= N^{PO}_{S} e_S \\
  N^{F}_{LS} &= N^{PO}_{LS} e_L e_S
  \label{eq:appendix:relationships}
\end{split}
\end{equation}

\noindent Expanding the first two of these equations out, we can see that

\begin{equation}
\begin{split}
  N^{F}_{10} + N^{F}_{11} &= ( N^{PO}_{10} + N^{PO}_{11} ) e_L \\
  N^{F}_{10} / e_L + N^{F}_{11} / e_L &= N^{PO}_{10} + N^{PO}_{11}
  \label{eq:appendix:leading_only}
\end{split}
\end{equation}

\noindent and

\begin{equation}
\begin{split}
  N^{F}_{01} + N^{F}_{11} &= ( N^{PO}_{01} + N^{PO}_{11} ) e_S \\
  N^{F}_{01} / e_S + N^{F}_{11} / e_S &= N^{PO}_{01} + N^{PO}_{11}
  \label{eq:appendix:subleading_only}
\end{split}
\end{equation}

\noindent Combining \EquationRef{eq:appendix:leading_only} and \EquationRef{eq:appendix:subleading_only}
we have

\begin{gather}
  \label{eq:appendix:final_PO_connection}
  N^{PO}_{10} + N^{PO}_{11} + N^{PO}_{01} + N^{PO}_{11} = N^{F}_{10} / e_L + N^{F}_{11} / e_L + N^{F}_{01} / e_S + N^{F}_{11} / e_S \\
  N^{PO}_{10} + N^{PO}_{11} + N^{PO}_{01} + (N^{F}_{11}/(e_L e_S)) = N^{F}_{10} / e_L + N^{F}_{01} / e_S + N^{F}_{11} ( 1 / e_L + 1 / e_S ) \\
  N^{PO}_{10} + N^{PO}_{01} + N^{PO}_{11}  = N^{F}_{10} / e_L + N^{F}_{01} / e_S + N^{F}_{11} ( 1 / e_L + 1 / e_S - 1/(e_L e_S) )
\end{gather}

\noindent so the sum $N^{PO}_{10} + N^{PO}_{01} + N^{PO}_{11}$ can be reconstructed
if events are weighted based on their final trigger category as shown in \TableRef{tab:appendix:trig_weights}.

\begin{table}
\begin{center}
  \begin{tabular}{ c c }
    Trigger Category & Required weight \\
    \midrule
    $T_{00}$         & 0               \\
    $T_{10}$         & $1/e_L$         \\ 
    $T_{01}$         & $1/e_S$         \\
    $T_{11}$         & $1/e_L/e_S$     \\       
  \end{tabular}
  \caption{Required weights for each trigger category necessary to allow reconstruction
           of the total number of events that would have been observed without trigger
           inefficiency.}
  \label{tab:appendix:trig_weights}
\end{center}
\end{table}

Note that this will not allow reconstruction of the numbers of events
passing each individual trigger category - it is only possible to obtain the total
number of events which would have passed the logical OR of these triggers. For this
reason, it is not possible to use the more statistically precise effective luminosity
correction discussed in \SectionRef{sec:forward-inclusive:transition_triggers}.
